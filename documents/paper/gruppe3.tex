% This is samplepaper.tex, a sample chapter demonstrating the
% LLNCS macro package for Springer Computer Science proceedings;
% Version 2.21 of 2022/01/12
%
\documentclass[runningheads]{llncs}
%
\usepackage[T1]{fontenc}
% T1 fonts will be used to generate the final print and online PDFs,
% so please use T1 fonts in your manuscript whenever possible.
% Other font encondings may result in incorrect characters.
%
\usepackage{graphicx}
% Used for displaying a sample figure. If possible, figure files should
% be included in EPS format.
%
% If you use the hyperref package, please uncomment the following two lines
% to display URLs in blue roman font according to Springer's eBook style:
%\usepackage{color}
%\renewcommand\UrlFont{\color{blue}\rmfamily}
%
\usepackage[ngerman]{babel}
\usepackage{wrapfig}
\begin{document}
%
\title{Gruppe 3: Catchphrase?}
%
%\titlerunning{Abbreviated paper title}
% If the paper title is too long for the running head, you can set
% an abbreviated paper title here
%
\author{...\inst{1}\and
...\inst{1}\and
...\inst{1}\and
...\inst{1}}
%
\authorrunning{F. Author et al.}
% First names are abbreviated in the running head.
% If there are more than two authors, 'et al.' is used.
%
\institute{FernUniversität in Hagen, Universitätsstraße 47, 58097 Hagen, Deutschland
\email{\{...\}@studium.fernuni-hagen.de}\\
\url{https://www.fernuni-hagen.de}}
%
\maketitle              % typeset the header of the contribution
%
%
\section{Einleitung}
TODO Gruppenbeschreibung
\section{Kommunikation und Kommunikationsmittel}
TODO Gruppentreffen, Github

\section{Technische Rahmenbedingungen und Softwarebasisarchitektur}
TODO Java, Bdi, 2 Agentensysteme, UML

\section{Gruppenbeitrag Heinz Stadler}
 
\subsection{Agent V1 - Architektur}
Aufbau

\subsection{Wissensverwaltung}
Belief

\subsection{Wegfindung}
Pathfinding

\subsection{Ziel- und Absichtsfindung}
Desires

\subsection{Verifikation und Problemfindung}
Tests / Debugger

\subsection{...}

\section{Gruppenbeitrag Melinda Betz}

\section{Gruppenbeitrag Phil Heger}

\section{Gruppenbeitrag Björn Wladasch}

\section{Turniere}
\subsubsection{Turnier 2}
\subsubsection{Turnier 3}
\subsubsection{Turnier 4}
\subsubsection{Turnier 5}
\subsubsection{Turnier 6}

\section{Rekapitulation und Ausblick}
Vor- und Nachteile der Entscheidung von zwei Architekturen
Was sollte noch verbessert werden
Wie sind wir zufrieden


%
% ---- Bibliography ----
%
% BibTeX users should specify bibliography style 'splncs04'.
% References will then be sorted and formatted in the correct style.
%
% \bibliographystyle{splncs04}
% \bibliography{mybibliography}
%
\begin{thebibliography}{8}
	\bibitem{Ahlbrecht2021}
	Ahlbrecht, T., Dix, J., Fiekas. N. und T. Krausburg: The Multi-Agent Programming Contest 2021, Springer, Heidelberg, 2021
	\bibitem{Hart1968}
	Hart, P. E., Nilsson, N. J. und Raphael, B.: A Formal Basis for the Heuristic Determination of Minimum Cost Paths, in IEEE Transactions on Systems Science and Cybernetics, 4. Auflage, Nummer 2, Seiten 100-107, Juli 1968
	\bibitem{Weiss2000}
	Weiss, G.: Multiagent Systems, 2. Auflage, The MIT Press, Cambridge, 2000
	\bibitem{EISMASSim}
	github.com/agentcontest/massim\_2022, agentcontest/massim\_2022, \\ https://github.com/agentcontest/massim\_2022/blob/main/docs/eismassim.md, EISMASSim Documentation, 21.08.2022
	\bibitem{Bratman1987}
	Bratman, M.: Intention, plans, and practical reason, Harvard University Press, Cambridge, 1987
\end{thebibliography}
\end{document}
